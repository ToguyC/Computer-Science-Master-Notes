I made this document for my own use, but I thought that typed notes might be of interest to others. There are mistakes, it is impossible not to make any. If you find some, please feel free to share them with me (grammatical and vocabulary errors are of course also welcome). You can contact me at the following e-mail address:
\begin{center}
    \texttt{joachim.favre@epfl.ch}
\end{center}

If you did not get this document through my GitHub repository, then you may be interested by the fact that I have one on which I put those typed notes and their \LaTeX{} code. Here is the link (make sure to read the README to understand how to download the files you're interested in):
\begin{center}
    \url{https://github.com/JoachimFavre/EPFLNotesIN}
\end{center}

Please note that the content does not belong to me. I have made some structural changes, reworded some parts, and added some personal notes; but the wording and explanations come mainly from the Professor, and from the book on which they based their course.

I think it is worth mentioning that in order to get these notes typed up, I took my notes in \LaTeX{} during the course, and then made some corrections. I do not think typing handwritten notes is doable in terms of the amount of work. To take notes in \LaTeX{}, I took my inspiration from the following link, written by Gilles Castel. If you want more details, feel free to contact me at my e-mail address, mentioned hereinabove.
\begin{center}
    \url{https://castel.dev/post/lecture-notes-1/}
\end{center}

I would also like to specify that the words ``trivial'' and ``simple'' do not have, in this course, the definition you find in a dictionary. We are at EPFL, nothing we do is trivial. Something trivial is something that a random person in the street would be able to do. In our context, understand these words more as ``simpler than the rest''. Also, it is okay if you take a while to understand something that is said to be trivial (especially as I love using this word everywhere hihi).

Since you are reading this, I will give you a little advice. Sleep is a much more powerful tool than you may imagine, so do not neglect a good night of sleep in favour of studying (especially the night before an exam). I wish you to have fun during your exams.
