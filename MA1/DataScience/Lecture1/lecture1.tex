% !TeX program = lualatex

\documentclass[a4paper]{article}

% Expanded on 2021-11-01 at 13:13:15.

\usepackage{../../style}

\begin{document}
	\lecture{1}{2024-09-17}{Concepts fondamentaux et applications}{
		\begin{itemize}[left=0pt]
			\item Explication de l'organisation du cours
			\item Algèbre linéaire
			\item Probabilités et statistiques
			\item Données à haute dimensionnalité
		\end{itemize}
	}

	\section{Organisation}
	\begin{parag}{Moodle}
		Toutes les informations sont sur le Moodle. À consulter régulièrement.
	\end{parag}
	
	\begin{parag}{Ouvrage de référence}
		\begin{itemize}
			\item \textbf{Foundations of Data Science} par Blum, Hopcroft, Kannan, Cambridge University Press, 2020. Disponible à la librairie.
			\item \textbf{Data Mining and Machine Learning: Fundamental Concepts and Algorithms} par Mohammed J. Zaki, Wagner Meira, Gr. Cambridge University Press, 2nd Ed., 2020. Disponible à la librairie.
		\end{itemize}
	\end{parag}

	\section{Algèbre linéaire}
	\begin{parag}{Concepts généraux}
		\begin{itemize}
			\item Qu'est-ce qu'un \important{vecteur / espace vectoriel} ?
			\item Qu'est ce que le \important{produit scalaire} représente ?
			\item Pourquoi la \important{projection} est un concept important ?
			\item Qu'est ce qui est critique par rapport au \important{age things} ?
		\end{itemize}
	\end{parag}

	\subsection{Vecteurs et espace vectoriel}
	\begin{parag}{Définition d'un espace vectoriel}
		Soit un champ $K(\mathbb{Q}\ \text{ou}\ \mathbb{C})$ et un ensemble $V$. Un espace vectoriel sur $K$, est un ensemble $E$, dont les éléments sont appelés vecteurs muni de deux lois :
		 \begin{itemize}
			 \item Une loi de composition interne \guillemotleft+\guillemotright : $E^2 \to E$, appelée addition ou somme vectorielle
			 \item Une loi de composition externe à gauche \guillemotleft\cdot\guillemotright : $K \times E \to E$, appelée multiplication par un scalaire
		\end{itemize}

		tel que les propriétés suivantes soient vérifiées :
		\begin{enumerate}
			\item ($E$, $+$) est un groupe abélien, autrement dit :
			\begin{itemize}
				\item la loi \guillemotleft+\guillemotright est commutative,
				\item elle est associative,
				\item elle admet un élément neutre $\vec{0_E}$ appelé vecteur nul et, 
				\item tout vecteur $v$ a un opposé, noté  $-v$.
			\end{itemize}
			C'est-à-dire que pour tous vecteurs $\vec{u}$, $\vec{v}$ et $\vec{w}$ de $E$:
			\begin{align}
			\begin{split}
				\vec{u} + \vec{v} &= \vec{v} + \vec{u} \\
				\vec{u} + (\vec{v} + \vec{w}) &= (\vec{u} + \vec{v}) + \vec{w} \\
				\vec{0_E} + \vec{v} &= \vec{v} \\
				\vec{u} + (-\vec{u}) &= \vec{0_E}
			\end{split}
			\end{align}
		\item La loi \guillemotleft\cdot\guillemotright vérifie les propriétés suivantes :
			\begin{itemize}
				\item elle est distributive à gauche par rapport à la loi \guillemotleft+\guille de $E$ et à droite par rapport à l'addition du corps  $K$,
				\item elle vérifie une associativité mixte (par rapport à la multiplication dans $K$),
				\item l'élément neutre multiplicatif du corps $K$, noté  $\vec{1_K}$, est neutre à gauche pour \cdot.
			\end{itemize}
			C'est-à-dire que pour tous vecteur $\vec{u}$,  $\vec{v}$ et  $\vec{w}$ de  $E$, et tous scalaire $\lambda$, $\mu$ :
			\begin{align}
			\begin{split}
				\lambda \cdot \vec{u} + \vec{v} &= (\lambda \cdot \vec{u}) + (\lambda \cdot \vec{v}) \\
				(\lambda + \mu) \cdot \vec{u} &= (\lambda \cdot \vec{u}) + (\mu \cdot \vec{u}) \\
				(\lambda\mu) \cdot \vec{u} &= \lambda \cdot (\mu \cdot \vec{u}) \\
				1_K \cdot \vec{u} &= \vec{u}
			\end{split}
			\end{align}
		\end{enumerate}

		Ces axiomes impliquent que $E$ est non vide et pour tout vecteur $\vec{u}$ de $E$ et tout scalaire $\lambda$ :
		\begin{align}
		\begin{split}
			\lambda \cdot \vec{u} &= \vec{0_E} \Leftrightarrow (\lambda = \vec{0_K}\ \text{ou}\ \vec{u} = \vec{0_E}) \\
			(-\lambda) \cdot \vec{u} &= -(\lambda \cdot \vec{u}) = \lambda \cdot (-\vec{u})
		\end{split}
		\end{align}
	\end{parag}

	\section{Représentation des espace à haute dimensionnalité}
	\begin{parag}{Flot de données et représentation}
		Le flot typique des données vient d'un capteur, qui produit des données brutes qui doivent ensuite être représentée d'une façon ou d'une autre.

		\imagehere{data-flow.png}
		
		\begin{center}
		\begin{tabular}{ | l | l | l | }
			\hline
			\textbf{Capteur} & \textbf{Données brutes} & \textbf{Représentation} \\
			\hline \hline
			Caméra & Pixels de l'image & Matrice \\
			Population & Résultats de vote & Matrice \\
			Document texte & Fréquence de mots & Matrice \\
			Réseaux sociaux & Relations & Matrice \\
			Environnement & Mesures & Matrice \\
			\hline
		\end{tabular}
		\end{center}

		On obtient donc des données $\mathcal{X} = \{x_1, \cdots, x_n\}$ où $x_i \in \Omega \subseteq \mathbb{R}^D \quad \forall i \in [\mathbb{N}]$.
	\end{parag}

	\begin{parag}{Vue statistique}
		$\mathcal{X}$ est un échantillon de l'ensemble de taille $N$ à $D$-dimensions où $x_i$ correspond à une mesure de la variable aléatoire associée (p.e. Température, humidité). Chaque $x_i$ peut être associé à une PDF, comme par exemple un lancé d'un dé-6 étant limité aux valeurs $[1, 6]$ où chaque face à une probabilité de $\frac{1}{6}$.

		Il est donc possible d'associé une probabilité d'apparition pour chaque valeur de l'ensemble de donnée.
	\end{parag}

	\begin{parag}{Vue algébrique}
		Soit $\mathcal{X}$, on forme une matrice $X = \left \begin{pmatrix} | && | \\ x_1 & \cdots & x_N \\ | && | \end{pmatrix} \right \in \mathbb{R}^{D \times N}$ dont les colonnes $X_{\cdot j} = x_j$ sont les vecteurs de données.
	\end{parag}

	\vspace{2cm}

	\emph{Le cours va porter sur les différentes vues possible (statistique et algébrique), et aussi comprendre ce qu'il se passe quand le nombre de dimension augmente ($D$ et $N$ dans $\mathbb{R}^{D \times N}$)}

	\emph{quand on a un nombre N fix (samples) et que l'on veut augmenter le nombre de bins, la stabilité de la représentation se dégradera de façon exponentielle (slide 9)}

	\emph{$D$ (nombre de dimension ou features) doit être très contrôlé, car si ce nombre augmenter, les données deviennent exponentiellement instable}

	KNN c'est le fait de choisir les K voisins les plus proches, $\epsilon$NN c'est le fait de choisir tout les voisin dans un périmètre $\epsilon$.
\end{document}
